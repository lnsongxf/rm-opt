
%\documentclass[12pt,handout]{beamer}
\documentclass[show notes]{beamer}%
\usepackage{amssymb}
\usepackage{amsmath}
\usepackage[labelformat=empty]{caption}
\usepackage{graphicx}
\usepackage[english]{babel}
\usepackage[latin1]{inputenc}
\usepackage[T1]{fontenc}
\usepackage{graphics}
%\usepackage{mathptmx}%
\setcounter{MaxMatrixCols}{30}%
\usepackage{amsfonts}
\usepackage{media9}
\usepackage{movie15}

\usepackage{pgfplots}
\pgfplotsset{compat=1.13}
% \usepackage{wrapfig}
\usepackage{marvosym}
\usepackage{listings}
\usepackage[framed,numbered]{matlab-prettifier}

\definecolor{dkgreen}{rgb}{0,0.6,0}
\definecolor{gray}{rgb}{0.5,0.5,0.5}




\providecommand{\U}[1]{\protect\rule{.1in}{.1in}}

\newenvironment{stepenumerate}{\begin{enumerate}[<+->]}{\end{enumerate}}
\newenvironment{stepitemize}{\begin{itemize}[<+->]}{\end{itemize} }
\newenvironment{stepenumeratewithalert}{\begin{enumerate}[<+-| alert@+>]}{\end{enumerate}}
\newenvironment{stepitemizewithalert}{\begin{itemize}[<+-| alert@+>]}{\end{itemize} }
%\usetheme{Szeged}
%\mode<presentation>
%{
%}
%\useoutertheme{default}
%\makeatletter

\definecolor{midgreen}{rgb}{0,0.7,0}
\definecolor{purple}{rgb}{0.5, 0.0, 0.5}
\definecolor{bluegreen}{rgb}{0.0,0.5, 0.5}
\definecolor{orange}{rgb}{.8,0.33, 0}
\definecolor{redgreen}{rgb}{0.5, 0.5, 0.0}
\definecolor{lightgray}{rgb}{0.8,0.8,0.8}

\def\blue{\color{blue}}
\def\red{\color{red}}
\def\green{\color{midgreen}}
\def\purple{\color{purple}}
\def\bluegreen{\color{bluegreen}}
\def\orange{\color{orange}}
\def\redgreen{\color{redgreen}}
\def\black{\color{black}}
\def\gray{\color{gray}}
\def\ltgray{\color{lightgray}}

\newcommand\V{\mathbb{V}}
\newcommand\UU{\mathbb{U}}
\newcommand\W{\mathbb{W}}
\newcommand\R{\mathbb{R}}
\newcommand\C{\mathbb{C}}
\newcommand\vv{\vec{v}}
\newcommand\vu{\vec{u}}
\newcommand\vw{\vec{w}}
\newcommand\vx{\vec{x}}
\newcommand\vy{\vec{y}}
\newcommand\vz{\vec{z}}
\newcommand\ve{\vec{e}}
\newcommand\vd{\vec{d}}
\newcommand\vc{\vec{c}}
\newcommand\va{\vec{a}}
\newcommand\vb{\vec{b}}
\newcommand\vzero{\vec{0}}
\newcommand\cB{\mathcal{B}}


\DeclareMathOperator{\Span}{Span}
\DeclareMathOperator{\Nul}{Nul}
\DeclareMathOperator{\Col}{Col}
\DeclareMathOperator{\Ker}{Ker}
\DeclareMathOperator{\Null}{Null}


\def\divideline{\line(1,0){352}}
\newcommand\be{\begin{enumerate}[(1)]}
	\newcommand\ee{\end{enumerate}}





\mode<presentation> {
	\usetheme{Szeged}
%		\usecolortheme{frigatebird}
	%	\usecolortheme{seahorse}
	\usecolortheme{beaver}
	%	\usecolortheme{whale}
	% \usefonttheme{serif}
	%\useoutertheme{shadow}
	\useoutertheme{infolines}
	\useoutertheme{miniframes} 
	%  \setbeamercovered{transparent}  
	\setbeamercovered{invisible}
}





\setbeamertemplate{footline}{
	\leavevmode%
	\hbox{
		\begin{beamercolorbox}[wd=.5\paperwidth,ht=2.25ex,dp=1ex,center]{title in head/foot}%
			\usebeamerfont{title in head/foot}\insertshorttitle		\end{beamercolorbox}%
		\begin{beamercolorbox}[wd=.5\paperwidth,ht=2.25ex,dp=1ex,right]{date in head/foot}%
			\usebeamerfont{date in head/foot}\insertshortdate{}\hspace*{2em}
			\insertframenumber{} / \inserttotalframenumber\hspace*{2ex} 
	\end{beamercolorbox}}%
	\vskip0pt%
}

\makeatother
\setbeamertemplate{navigation symbols}{}
%BeginMSIPreambleData
\ifx\pdfoutput\relax\let\pdfoutput=\undefined\fi
\newcount\msipdfoutput
\ifx\pdfoutput\undefined\else
\ifcase\pdfoutput\else
\msipdfoutput=1
\ifx\paperwidth\undefined\else
\ifdim\paperheight=0pt\relax\else\pdfpageheight\paperheight\fi
\ifdim\paperwidth=0pt\relax\else\pdfpagewidth\paperwidth\fi
\fi\fi\fi




\title[Solving economic problems with MATLAB]{Solving economic problems with MATLAB}

\author[Antonio Mele]{Antonio Mele}
\institute[University of Surrey]{University of Surrey}
\date{}
%\date[October 21, 2010]{October 21, 2010}

\begin{document}


\lstset{ style = Matlab-editor,  basicstyle = \mlttfamily, tabsize = 2 , morekeywords={bsxfun, fmincon, fminunc, fsolve, fzero,optimset}}

\lstnewenvironment{MATLAB}[1][,]{\lstset{language = Matlab-pretty, style = Matlab-editor,  basicstyle = \mlttfamily, morekeywords={parfor}, float,#1, }}{}


\newcommand{\inlineMATLAB}[1][,]{\lstinline[language = Matlab-pretty, style = Matlab-editor,  basicstyle = \mlttfamily, morekeywords={parfor},#1,]}

%\lstset{language=Matlab,
%   keywords={break,case,catch,continue,else,elseif,end,for,function,
%      global,if,otherwise,persistent,return,switch,try,while},
%   basicstyle=\ttfamily,
%   keywordstyle=\color{blue},
%   commentstyle=\color{dkgreen},
%   stringstyle=\color{red},
%   numbers=left,
%   numberstyle=\tiny\color{gray},
%   stepnumber=1,
%   numbersep=10pt,
%   backgroundcolor=\color{white},
%   tabsize=4,
%   showspaces=false,
%   showstringspaces=false,
%   lineskip={-.02\baselineskip}}

\newtheorem{res}{\sc Result}



\begin{frame}
  \titlepage  
\end{frame}


\section{Introduction}



%\begin{frame}
%	\begin{center}
%		\begin{figure}[ht]
%			\includemedia[
%			width=\linewidth,
%			%keepaspectratio,
%			height=0.6\linewidth,
%			activate=pageopen,  
%			addresource=MatchmakerNumMethods.wmv,
%			flashvars={
%				source=MatchmakerNumMethods.wmv
%				&loop=false  % loop the video
%				&playbutton= fancy 
%			}
%			]{}{VPlayer.swf}
%		\end{figure}
%	\end{center}
%\end{frame}

%\begin{frame}
%	\begin{center}
%		\begin{figure}[ht]
%			\includemovie[poster,autoplay, mouse=true]{12cm}{9cm}{MatchmakerNumMethods.wmv}
%		\end{figure}
%	\end{center}
%\end{frame}

\begin{frame}[fragile]{Why numerical methods?},
\begin{itemize}[<+->]
\item Many economic problems are VERY complicated
\item No analytical solution
\item Often, even if we can describe some QUALITATIVE features, we need numerical methods for QUANTITATIVE results
\item Mostly: solving non-linear equations and optimisation problems
\end{itemize}
\end{frame}

\section{Nonlinear equations}


\begin{frame}[fragile]
\frametitle{Optimization Toolbox: basics}
The toolbox includes:
\begin{itemize}[<+->]
\item linear programming
\item quadratic programming
\item binary integer programming
\item nonlinear optimization
\item nonlinear least squares
\item systems of nonlinear equations
\item multiobjective optimization
\end{itemize}

\end{frame}


\begin{frame}[fragile]
\frametitle{Optimization Toolbox: basics (cont.)}
Solvers are minimizers!\pause

If you perform a \textbf{maximization}: define your objective as $-f(x)$ and remember to change sign to the value obtained.\pause

All the functions are \textbf{function functions}: they take other functions as inputs. In particular:
\begin{itemize}
\item the objective function is a \textbf{function} file
\item nonlinear constraints must be set in a \textbf{function} file
\end{itemize}\pause

Options: set with \texttt{optimset}

\end{frame}


\begin{frame}[fragile]
\frametitle{REMINDER: what is a function?}


\begin{lstlisting}
    function z = olscoefficient(X,Y)    
    	z = inv(X'*X)*(X'*Y);
    end
\end{lstlisting}


\end{frame}




\begin{frame}[fragile]
\frametitle{Nonlinear equations in one variable}
Use the function \texttt{fzero}: finds roots of continuous functions

Syntax:

\begin{lstlisting}
    [x, fval] = fzero('objfun',x0);
\end{lstlisting}

\texttt{x}: optimum

\texttt{fval}: value of the objective function calculated in the optimum

\texttt{objfun}: function file where we have stored the objective function

\texttt{x0}: initial condition from which \texttt{fzero} looks for a solution


\end{frame}



\begin{frame}[fragile]
\frametitle{Nonlinear equations in more than one variable}

If you have $n$ nonlinear equations $F_i (x) = 0$, with $x \in R^n$, use \texttt{fsolve}.

Syntax:

\begin{lstlisting}
    [x, fval] = fsolve('objfun',x0);
\end{lstlisting}

\texttt{x}: optimum

\texttt{fval}: value of the objective function calculated in the optimum

\texttt{objfun}: function file where we have stored the objective function

\texttt{x0}: initial condition from which \texttt{fsolve} looks for a solution

\end{frame}



\begin{frame}[fragile]{Example 1}
Solve the following system of equations:
\begin{align*}
& c_1^{-\sigma} = \beta (1+r) c_2^{-\sigma} \\
& c_1 + \frac{c_2}{1+r} = y_1 + \frac{y_2}{1+r} 
\end{align*}
where $r=0.05$, $\sigma = 2$, $\beta = .99$, and $y_1 = y_2 = 1$.
\end{frame}


\begin{frame}[fragile]{Example 2}
Same as before, but with an initial amount of savings $s_0$:
\begin{align*}
& c_1^{-\sigma} = \beta (1+r) c_2^{-\sigma} \\
& c_1 + \frac{c_2}{1+r} = y_1 + \frac{y_2}{1+r} + s_0
\end{align*}
where $r=0.05$, $\sigma = 2$, $\beta = .99$, and $y_1 = y_2 = 1$.

Solve for the optimal allocation for different values of the initial savings. 
\end{frame}


\section{Optimization}


\begin{frame}[fragile]
\frametitle{Optimization: general rules}

Types of constraints:  \pause

\begin{enumerate}

\item \textbf{Bound Constraints}: $x \geq l$, $x \leq u$ \pause

\item \textbf{Linear Inequality Constraints}: $A \cdot x \leq  b$, where A is an m by n matrix, which represents m constraints for an n dimensional vector x, and b is m dimensional \pause

\item \textbf{Linear Equality Constraints}: $Aeq \cdot x = beq$, same dimensionality of linear inequality constraints \pause

\item \textbf{Nonlinear Constraints}: $c\left(x \right) \leq 0$ and $ceq\left(x\right) = 0$. Both $c$ and $ceq$ are scalars or vectors representing several constraints

\end{enumerate}

\end{frame}



\begin{frame}[fragile]
\frametitle{Setting options}

Each algorithm has many options on the type of algorithm to use, on the output to show in command window, the convergence criterion, etc. \pause

To set them: use \texttt{optimset}
{\small
\begin{lstlisting}
options = optimset('param1',value1, 'param2',value2,...);
\end{lstlisting}
}\pause

\textbf{IMPORTANT}: some parameter values are strings, therefore you have to enter them between '    '. Example:

\begin{lstlisting}
options  = optimset('Display','iter');
\end{lstlisting}

\end{frame}




\begin{frame}[fragile]
\frametitle{Unconstrained Minimization}
No constraints
\begin{equation*}
\min_{x} f(x)
\end{equation*} \pause

Use \texttt{fminunc}:

\begin{lstlisting}
[x, fval] = fminunc('objfun',x0)
\end{lstlisting}
\end{frame}

\begin{frame}[fragile]
\frametitle{Constrained Minimization}

\begin{align*}
\min_{x} &\quad f(x) \\
s.t. \quad & x \geq LB, \quad x \leq UB \\
& A \cdot x \leq B, \quad Aeq \cdot x = Beq \\
& c(x) \leq 0, \quad ceq(x) = 0
\end{align*} \pause
Use \texttt{fmincon}:

\begin{lstlisting}
[x, fval] = fmincon('objfun',x0,A,B,Aeq,Beq,...
    LB,UB,nonlcon);
\end{lstlisting}
When one or more constraints absent: use \texttt{[]}

\end{frame}


\begin{frame}[fragile]
\frametitle{Writing a nonlinear constraint function} \pause
It must have a particular structure

\begin{lstlisting}
function [c, ceq] = nonlinconst(input1,input2,...)

c(1) = ...
c(2) = ...
...
ceq(1) = ...
ceq(2) = ...
...
\end{lstlisting} \pause
If no constraints of one type: use \texttt{ceq = [];}

\end{frame}

\begin{frame}[fragile]{Example 1}
Use fmincon to maximize the utility function $u(c) = \frac{c^{1-\sigma}}{1-\sigma}$ under the constraints:
    \begin{align*}
    & c \geq 0 \\
    & c \leq y
    \end{align*}
    where $\sigma = 2$ and $y = 1$.
\end{frame}




\begin{frame}[fragile]{Example 2}
Take again a two-periods economy with an initial amount of savings $s_0$. Your problem is
\begin{align*}
 \max_{c_1,c_2} \, & \frac{c_1^{1-\sigma}}{1-\sigma} + \beta \frac{c_2^{1-\sigma}}{1-\sigma} \\
& s.t. \quad  c_1 + \frac{c_2}{1+r} \leq y_1 + \frac{y_2}{1+r} + s_0
\end{align*}
where $r=0.05$, $\sigma = 2$, $\beta = .99$, and $y_1 = y_2 = 1$. Solve for the optimal allocation for different values of the initial savings. (Hint: vectorizing the procedure is not possible here, therefore you need to use a \texttt{for} loop).
\end{frame}


\begin{frame}[fragile]{Example 3}
Maximize the utility function $u(c_1,...,c_{10}) =\sum_{i=1}^{10} \frac{ c_i^{1-\sigma}}{1-\sigma}$ under the constraints:
    \begin{align*}
    & c_i \geq 0 \text{ for all } i \\
    & \sum_{i=1}^{10} c_i \leq  y \\
    & 2c_3 + c1 = 12 \\
    & 0.5 ( c_{5} - c_4)^{2} = 4
    \end{align*}
    where $\sigma = 2$ and $y = 100$.
\end{frame}




\end{document}
